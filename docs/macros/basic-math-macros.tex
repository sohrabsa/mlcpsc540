\newcommand\bracearray[1]{\left\{ \begin{array}{l} #1 \end{array} \right.}
\newcommand\bracearraypair[2]{\bracearray{#1 \\ #2}}
\newcommand\bracearraytriple[3]{\bracearray{#1 \\ #2 \\ #3}}
\newcommand\bracearrayquad[4]{\bracearray{#1 \\ #2 \\ #3 \\ #4}}
\newcommand\dotprod[2]{\langle #1, #2\rangle}


\newcommand\bracearraycond[1]{\left\{ \begin{array}{ll} #1 \end{array} \right.}

\newcommand\partialderiv{\frac{\partial}{\partial \theta_i}}

\def\Q{{\mathbb Q}}        % rationals
\def\Z{{\mathbb Z}}        % integers
\def\N{{\mathbb N}}        % naturals
\def\M{{\mathbb M}}        % naturals
\def\R{{\mathbb R}}        % reals
\def\C{{\mathbb C}}      % complex
\def\Cf{{\mathbf C}}     % continuous functions on (.)

\def\P{{\mathbb P}}        % probability
\def\E{{\mathbb E}}        % expectation 
\def\1{{\mathbf 1}}        % indicator
\def\var{{\mathop{\mathbf Var}}}    % variance

\def\I{{\mathcal I}} % index set 

\def\F{{\cal F}} % sigma-algebra

\def\L{{\mathbf L}}     % L, as in L^2

\def\ascv{\stackrel{\scriptscriptstyle a.s.}{\longrightarrow}}     % almost sure convergnece
\def\pcv{\stackrel{\scriptscriptstyle \P}{\longrightarrow}}        % convergence in P
\def\ltcv{\stackrel{\scriptscriptstyle\L^2}{\longrightarrow}}      % L2 convergnece
\def\lpcv{\stackrel{\scriptscriptstyle\L^p}{\longrightarrow}}      % Lp convergnece
\def\dcv{\stackrel{\scriptscriptstyle d}{\longrightarrow}}         % convergence in d
\def\deq{\stackrel{\scriptscriptstyle d}{=}}         % equal in d
\def\iidsim{\stackrel{\scriptscriptstyle \textrm{iid}}{\sim}}         % iid
\def\toinf{\to \infty}

\def\ci{\perp\!\!\!\perp}  % conditional independence



% phylo macros
\def\ver{{\mathscr V}}
\def\edg{{\mathscr E}}
\def\leaf{{\mathscr L}}
\def\cdata{{\mathcal Y}}
\def\edata{{\mathcal E}}
\newcommand{\pa}[1]{\mathrm{pa}\left(#1\right)}
\newcommand{\branch}[1]{\mathrm{b}\left(#1\right)}

% distribution macros
\newcommand{\bbinom}[1]{\mathrm{BetaBinomial}\left(#1\right)}
\newcommand{\gammaDist}[1]{\mathrm{Gamma}\left(#1\right)}


\DeclareMathOperator*{\argmin}{argmin}
\DeclareMathOperator*{\argmax}{argmax}

\newcommand\transp[1]{{#1}^{\textrm{T}}}


\newcommand{\ud}{\,\mathrm{d}}

\DeclareMathOperator{\dom}{dom}


\newtheorem{theorem}{Theorem}
\newtheorem{lemma}[theorem]{Lemma}
\newtheorem{proposition}[theorem]{Proposition}
\newtheorem{assumption}[theorem]{Assumption}
\newtheorem{claim}[theorem]{Claim}
\newtheorem{corollary}[theorem]{Corollary}
\newtheorem{definition}[theorem]{Definition}
% \newenvironment{proof}{{\bf Proof:}}{\hfill\rule{2mm}{2mm}}
\newenvironment{proofsketch}{{\bf Proof Sketch:}}{\hfill\rule{2mm}{2mm}}


